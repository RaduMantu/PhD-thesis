% This file contains the abstract of the thesis

In this thesis we challenge the current consensus on how application visibility
is achieved on Next Generation Firewalls. For the better part of the past two
decades, firewalls have been relying on Deep Packet Inspection (DPI) for this purpose.
However, this approach has a number of significant downsides. First of all,
user privacy is compromised since DPI can only be applied on unencrypted traffic.
Second, the traffic that is generated by an application is not necessarily
indicative of its version, nor does it have the capability of disclosing what
libraries are being used by said application. Thus, application visibility can only
be used to restrict the use of certain functions (e.g., file transfers) but does
not guarantee that a vulnerable or potentially compromised process cannot access
the network.

Our solution to this problem consists of a distributed firewall. By leveraging
Operating System level knowledge on each host, we can prevent processes
that have known vulnerable binaries mapped in their virtual address space from
emitting or receiving network traffic. Furthermore, we are able to annotate
packets that comply with the local firewall policy for future verification on
middleboxes. In order to develop this annotation scheme, we have performed
a number of experiments in public networks that allowed us to ascertain which
network protocol extension mechanism is most suited for this task. In order to
demonstrate the portability of our solution, we have implemented multiple
prototypes. \daf{} is the user space variant of our distributed firewall and is
the most feature complete. \scout{} is an alternative that is implemented as
a kernel module and demonstrates greatly improved performance. Additionally, we
preset a proof of concept annotator based on the Windows Filter Engine. Finally,
we offer plugins for \texttt{iptables} and \texttt{snort3} that show how our
annotation scheme can seamlessly integrate with existing software stacks.

