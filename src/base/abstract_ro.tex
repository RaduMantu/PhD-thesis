În această teză contestăm consensul actual privind modul în care se realizează
vizibilitatea aplicațiilor pe firewall-urile de nouă generație. Pentru cea mai
mare parte a ultimelor două decenii, firewall-urile au s-au bazat pe Deep Packet
Inspection (DPI) pentru a obține acest rezultat. Cu toate acestea, această abordare
are o serie de dezavantaje semnificative. În primul rând, confidențialitatea
utilizatorului este compromisă, deoarece DPI poate fi aplicată doar traficului
necriptat. În al doilea rând, traficul generat de o aplicație nu indică neapărat
versiunea acesteia și nici nu are capacitatea de a dezvălui ce biblioteci sunt
utilizate de respectiva aplicație. Astfel, vizibilitatea aplicației poate
fi utilizată doar pentru a restricționa utilizarea anumitor funcții implementate
aceasta (e.g., transferul de fișiere), dar nu garantează că un proces vulnerabil
sau potențial compromis nu poate accesa rețeaua.

Soluția noastră la această problemă constă într-un firewall distribuit. Prin
valorificarea cunoștințelor Sistemului de Operare la nivelul fiecărei gazde,
putem împiedica procesele care au binare vulnerabile cunoscute mapate în spațiul
lor de adresă virtuala să emită sau să primească trafic de pe rețea. În plus,
putem adnota pachetele care respectă politica firewall-ului local pentru validarea
lor pe middlebox-uri. Pentru a dezvolta această schemă de adnotare, am efectuat
o serie de experimente în rețele publice care ne-au permis să stabilim care
mecanismu de extindere a protocalelor de rețea este cel mai potrivit pentru
această sarcină. Pentru a demonstra portabilitatea soluției noastre, am
implementat mai multe prototipuri. \daf{} este varianta ce rulează strict în
user space și are cea mai completă funcționalitate. \scout{} este o alternativă
care este implementată ca un modul de kernel și demonstrează performanță mult
îmbunătățită. În plus, propunem un sistem de adnotare Proof of Concept bazat pe
Windows Filter Engine. În cele din urmă, oferim plugin-uri pentru \texttt{iptables}
și \texttt{snort3} care demonstrează cum sistemul nostru de annotare se poate
integra fără dificultate în stivele software existente.

