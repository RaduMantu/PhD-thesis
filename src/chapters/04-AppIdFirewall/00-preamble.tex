In 2003, Gartner Research marked the transition from classic firewalls to what they coined as Next Generation Firewalls (NGFW) \cite{stiennon2003paths}. Prior to this, firewalls used to analyze just the network and transport layer headers. This was sufficient due to a strong correlation between port numbers and services. However, the rapid diversification of said services and web applications gave rise to a new class of more potent adversaries that the previous generation of stateful firewalls could not contend with. As a result, NGFWs were developed for the purpose of achieving application awareness through deep packet inspection \cite{el2017survey}.

Although current firewalling technologies need to contend with new challenges such as micro-segmentation \cite{sheikh2021zero} or the advancing dissolution of network demarcations, the core tenets established by Gartner back in 2003 still hold true today. However, a concerted effort dating back over ten years (e.g., Google site ranking, Let's Encrypt, etc.) rapidly lead to the widespread adoption of HTTPS in the Internet. While undoubtedly a change for the better, traffic encryption has the downside of rendering deep packet inspection ineffective.

Unfortunately, efforts towards obtaining an encrypted traffic classifier have not yet yielded a sufficiently accurate solution \cite{husak2016https}. In lieu of better alternatives, firewall manufacturers decided to adopt SSL/TLS decryption \cite{radivilova2018decrypting}, thus sacrificing user privacy in exchange for overall network security. For outgoing connections, this is achieved via a forward proxy. Each new TLS connection to an external server can be intercepted by the firewall. Next, the firewall would emit a certificate with the Distinguished Name of the external server but signed by the Certification Authority (CA) of the local organization. Since this CA is configured a priori on all internal network hosts, the TLS connection succeeds and the firewall can access the plaintext application data, then forward it to the intended endpoint via a separate TLS connection of its own. For incoming connections, the firewall is assumed to have access to the private key used by the internal server for decryption.

Although there are heuristic-based approaches such as Cisco's Encrypted Traffic Analytics (ETA) \cite{manning2020aceta} or Palo Alto's App-ID \cite{malmgren2016comparative} that rely on metrics such as packet length, packet source or transmission rate, these are usually applied when encountering proprietary encryption protocols (i.e., protocols other than SSL/TLS or SSH). Furthermore, application identification on plaintext traffic is usually achieved via traffic fingerprinting. Both Cisco's Network-Based Application Recognition (NBAR) and App-ID compare the payload against a database of application-specific features called signatures.

