Application fingerprinting based on network traffic is a crucial requirement for correctly managing computer networks and enforcing security policies within them. In the early days of the internet, the Layer 4 ports were generally indicative of the application that was bound to them. However, the rapid maturation of these technologies has created uncertainty and lead to new, telemetry-based solutions \cite{bernaille2006early} to emerge. Being a long-term goal of companies such as Cisco that have had a vested interested for well over a decade \cite{zander2005automated}, machine learning techniques have recently become more prevalent in the identification of endpoint applications \cite{yamansavascilar2017application,guerra2022datasets}. Nonetheless, these technologies are yet to see large scale adoption in firewalling solutions due to inherent limitations, stemming in large part from the constantly accelerating diversification of web applications.

In this chapter we propose a different approach to ensuring network security: the distribution of responsibility \cite{ioannidis2000implementing} to multiple hosts in the network. Instead of centralizing the firewalling mechanism and compensating with increasingly costly and potentially inaccurate fingerprinting techniques, we rely on the operating system of each host to capitalize on its extensive control over the endpoint processes in order to correctly identify the afferent application. Additionally, we attach a proof of compliance to each emitted packet for other devices to verify.

This proof would demonstrate to other networked entities that each packet that originated from a certain host had been analyzed by an instance of our firewall. Moreover, we do not make the assumption that all potential endhosts and middlebox devices are aware of our annotation scheme. While some prior solutions \cite{parno2012using,zeldovich2008securing} have appended their own version of a proof of compliance to the payload, we argue that this would most assuredly break the application-level communication between incompatible devices. We emphasize this problem due to recent developments in corporate network infrastructure \cite{ward2014beyondcorp,osborn2016beyondcorp} where devices tend to seamlessly transition between networks. Without a guarantee that all guest devices would at least have knowledge of our annotation scheme, we decided to ere on the side of caution and only use the formal protocol extension mechanisms that were already in place (i.e., protocol options). This approach would allow unknown protocol options to be quietly ignored by the network stacks, so long as they correctly implement the afferent standards.

