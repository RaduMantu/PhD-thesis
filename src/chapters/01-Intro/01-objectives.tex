\section{Thesis Objectives}
\label{intro:objectives}

In our work, we revisit a concept briefly explored at the beginning of the 2000s,
namely Distributed Firewalls \cite{ioannidis2000implementing}. At the time, this
design posed numerous difficulties in terms of scalability, management, and
network integration. As a result, the industry distanced itself from this
approach and instead adopted the monolithic firewall architecture. In recent
years however, this model has become too restrictive. Where VLANs were once
sufficient for network segmentation, micro-segmentation is now steadily gaining
traction. Where internal networks were previously considered homogenous, mobile
devices are now free to transition between one another. As a response to these
changes, we have observed renewed efforts \cite{bringhenti2022optimizing,
tudosi2022secure, bagheri2020dynamic} towards solving those initial challenges
that discouraged the adoption of the distributed firewall paradigm.

We believe that distributed firewalling has the potential of offering an alternative
to application identification based on TLS decryption. Our aim is to demonstrate
that similar security guarantees can be provided without compromising user
privacy. To this end, we enumerate the following objectives that we aim to
accomplish in our work:

\begin{enumerate}
    \item \textbf{Study existing technologies used in ensuring network security:}
          Assess their utility, limitations and development challenges.
    \item \textbf{Implement a distributed firewall:} This firewall must be able
          to determine process identity by leveraging local operating system
          level knowledge, and allow rule configuration based on this information.
    \item \textbf{Identify a method of conveying network traffic compliance with
          filtration rules:} Any endpoint host or middlebox must be able to
          recognize traffic that has been verified on egress from its source
          system. Proof of comliance must be attached to each packet.
    \item \textbf{Explore methods of network integration:} Both our firewall and
          traffic annotation method must be easy to adopt and must not interfere
          with existing communication software stacks.
    \item \textbf{Assess development difficulty based on target operating system:}
          Although most servers and middleboxes are running Linux-based
          distributions, most users utilize Windows. In order to achieve
          standardization in IT networks, both will need to be supported to
          some degree.
   \item \textbf{Evaluate performance overhead in realistic environments:}
          Determine what are the resource requirements for running the firewall
          on a given system and what impact its integration will have on the
          network as a whole. Identify compromises that can be made in favor of
          performance.
\end{enumerate}

