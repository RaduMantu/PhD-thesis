\chapter{Introduction}
\label{intro:chapter}

During the early days of the Internet, external network threats were easy to counteract due to the unambiguous nature of clients, as well as the limited availability of web applications. The association between server port numbers and the listening applications could be drawn more reliably. In turn, this simplified assessing the purpose behind each new connection. In 2003 however, Gartner Research defined what they considered to be four sound approaches to gateway security, thus coining the term "Next Generation Firewall" (NGFW). With the purpose of opposing emerging network security threats, NGFWs integrated Intrusion Detection and Prevention Systems (IDS/IPS) and application awareness into conventional firewalls. Even at that time, it could be stated that endpoint applications were by and large detached from the meaning that the port they were bound to conferred in years prior. Thus, techniques such as JA3 or Markov Chain Fingerprinting have since been developed for the purpose of active threat mitigation. However, their efficacy (as well as that of IDS/IPS) is significantly diminished when applied to encrypted traffic. Despite ongoing efforts towards encrypted traffic classification, TLS traffic decryption unequivocally remains the most preponderant solution to this problem, levying a compromise between user privacy and general network security.

In spite of this shortcoming, NGFWs are more relevant than ever due to the ongoing adoption of Zero Trust network architectures. As enterprises seek to implement fine-grained, least-privilege resource access policies, the utilization of micro-segmentation has become increasingly common, as shown in a recent report on Alphabet's transition to the Zero Trust paradigm, i.e. Google BeyondCorp. We argue that current access decision systems which are commonly based on user authentication are insufficient for preventing attacks that are facilitated by vulnerable versions of common tools or shared libraries, e.g. the recent \texttt{liblzma} backdoor or software supply chain attacks. This problem is further exacerbated by the increasing fluidization of network demarcations, a phenomenon determined by the necessity of free movement between enterprise-owned and public networks. We believe that meeting these challenges should be aided by end-user devices.

\section{Thesis Contributions}

In this thesis, we provide a comprehensive solution to application awareness
in the form of a distributed firewall. This firewall has been implemented from
scratch and has two additional variants that are more limited in scope but serve
as proof of concepts for future efforts. Additionally, we propose a packet
annotation scheme that allows our firewall to convey proof of compliance for
emitted traffic to other instances of the same firewall or Intrusion Detection
and Prevention Systems or existing software firewalls.

We claim the following contributions:

\begin{enumerate}
    \item \textbf{Evaluation of protocol option acceptance in the Internet:}
          By implementing a traffic annotation tool with IP, TCP and UDP options,
          we perform reachability tests between 21 sites across the globe, and
          four cloud providers as well as our university.
    \item \textbf{Implementation of an application-aware distributed firewall:}
          We offer multiple variants of this firewall. The primary version is
          developed for Linux and runs strictly in userspace. A kernel-based
          alternative offers fewer features but demonstrates significant
          performance increase. The third variant represents a proof of concept
          implementation for Windows systems.
    \item \textbf{Evaluation in realistic environments:} We test our Linux-based
          firewall implementations in both datacenter environments, on servers
          with 10Gbps NICs, as well as desktop environments, on Intel NUC
          systems with 1Gbps NICs. We offer the user the option of tuning the
          firewall and select what security compromises are tolerable for the
          purpose of increasing performance.
    \item \textbf{Packet annotation scheme and compliance verification:} We
          propose a packet annotation scheme based on protocol options. Although
          we eventually decided on IP options as a conduit for the proof of
          compliance attached to each packet, we discuss the current limitations
          of TCP and UDP options that prevented us from implementing a Transport
          Layer alternative, as well as the current efforts of IETF working
          groups to remedy them.
    \item \textbf{Network integration solutions:} In order to integrate our
          packet annotation scheme in existing networks without the need to
          deploy our firewall on more hosts than necessary, we have implemented
          \texttt{iptables} and \texttt{snort3} modules that are able to
          verify the proof of compliance attached to each packet.
    \item \textbf{Overview of the state of the art in Network Fuzzing:}
          We explore the possibility of employing fuzzing as an additional
          measure for ensuring the security of the applications that we
          whitelist in our firewall policies.
\end{enumerate}

