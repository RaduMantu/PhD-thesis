\section{Thesis Structure}
\label{intro:structure}

This thesis is structured as follows:

In Chapter \ref{background:chapter} we discuss a number of concepts and
technologies that are referenced in this thesis. Section \ref{background:firewall}
presents technical information regarding the Netfilter framework and
\texttt{Netfilter Queue} in particular. This represents the basis for most
Intrusion Detection and Prevention Systems are implemented on Linux. Additionally,
we introduce concepts employed in state of the art, industrial grade firewalls.
Because this technology is often proprietary and the code is closed-source, we
mainly reference white papers and technical reports. Section \ref{background:protocols}
describes the logic behind the Layer 3 and Layer 4 protocol extension mechanisms
we took under consideration for our annotation scheme. Section \ref{background:difc}
offers a brief introduction to Distributed Information Flow Control and
distinguishes the annotation mechanisms specific to this type of systems from
ours.

Chapter \ref{extend:chapter} presents our efforts toward making an informed
selection in terms of a conduit for traffic annotation. In Section \ref{extend:ops}
we present our traffic annotator, a user space application capable of intercepting
packets and annotating them with IP, TCP or UDP options. These options are
specified by the user as a sequence of codepoints that are expanded in an
arbitrary manner. In addition to this tool, we offer an analysis of our protocol
options acceptance excepriments. This analysis includes reachability tests between
multiple cloud hosted instances located at different geographical locations,
across multiple cloud providers. For the purpose of reproducibility, we include
our experiment orchestration and analysis tools. Section \ref{extend:netfuzz}
represents a state of the art overview regarding network fuzzing. Because our
firewall is reponsible only for enforcing filtration policies based on application
identity, monitoring the internal state of running processes for the purpose of
anomaly detection falls outside its purview. Consequently, we asses the viability
of fuzzing network applications as a method of ensuring the soundness of any
whitelisted application prior to its deployment in live environments. This
section constitutes a preliminary overview. We discuss our future plans in
Chapter \ref{conclusion:chapter}.

In Chapter \ref{appfw:chapter} we present implementation details for the two
main variants of our distributed application firewall. Section \ref{appfw:daf}
contains the design, implementation details and evaluation of \daf{}, our most
feature-complete prototype. \daf{} runs exclusively in userspace (with the
exception of a few eBPF syscall probes) and makes use of \texttt{Netfilter Queue}
in a manner similar to \texttt{snort3} in intrusion prevention mode. In this
section we discuss the technical limitations of the available Linux kernel
interfaces for the purpose of deep packet inspection and cross-namespace
network management. Additionally, we offer a comprehensive analysis of the
impact \texttt{Netfilter Queue} has on performance, methods of handling its
idiosyncrasies, and design limitations that would need to be addressed in a
overhaul of the entire system. Section \ref{appfw:appscout} introduces \scout{},
a partial reimplementation of \daf{} as a kernel module. The reason why \daf{}
was designed to run exclusively in user space was for ease of integration
with existing systems, even those whose kernel was under lockdown (i.e., could
not be modified by the insertion of modules after boot). After assessing the
challenges in further optimizing \daf{}, we decided to relax this constraint
and permit the use of kernel modules without requiring the recompilation of the
kernel itself. Consequently, our alternative solution makes use of kprobes to
perform dynamic instrumentation on certain common paths of system call handlers
(e.g.: that of \texttt{mmap()} and friends). Under this model, the filtration
mechanism is no longer implemented in user space, but instead as an \texttt{iptables}
plugin, in kernel space.

In Chapter \ref{sign:chapter} se describe our packet annotation scheme. This
annotation scheme has the purpose of attaching proof of compliance with the
distributed firewall policy to individual packets. Alternatively, the limited
protocol header space can be used to provide middleboxes or conventional
(programmable) firewalls with information regarding the originating application.
Section \ref{sign:linux} focuses on the implementation of this feature as part
of \daf{}. In addition to a comprehensive description of the packet interception
and modification mechanism, we also describe the integration of our annotation
scheme with \texttt{iptables} and \texttt{snort3} for verification purposes.
Section \ref{sign:windows} contains a proof of concept implementation of the
annotation scheme on Windows, using the Windows Filter Engine. A notable
difference between the Linux and Windows implementations is that the former
embeds the tags within the IP header, while the latter does so within the TPC
header, as an experimental TCP option. This decision was made in order to ensure
that the Windows TCP/IP stack does not drop the traffic due to lack of support
for IP options. The \texttt{iptables} module used for validation was adapted to
work with TCP options instead of IP.

Chapter \ref{conclusion:chapter} concludes this thesis. Section
\ref{conclusion:summary} provides a summary of our work. Section
\ref{conclusion:contrib} reiterates our main contributions. Section
\ref{conclusion:future} outlines our immediate future goals, based on the
outcome of this thesis. Section \ref{publications:chapter} contains a list of
our publications that support the claims of this thesis.

