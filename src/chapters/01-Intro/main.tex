% vim: set tw=78 tabstop=4 shiftwidth=4 aw ai:

\chapter{Introduction}
\label{intro:chapter}

During the early days of the Internet, external network threats were easy to counteract due to the unambiguous nature of clients, as well as the limited availability of web applications. The association between server port numbers and the listening applications could be drawn more reliably. In turn, this simplified assessing the purpose behind each new connection. In 2003 however, Gartner Research defined what they considered to be four sound approaches to gateway security \cite{stiennon2003paths}, thus coining the term "Next Generation Firewall" (NGFW). With the purpose of opposing emerging network security threats, NGFWs integrated Intrusion Detection and Prevention Systems (IDS/IPS) and application awareness \cite{heino2022study} into conventional firewalls. Even at that time, it could be stated that endpoint applications were by and large detached from the meaning that the port they were bound to conferred in years prior. Thus, techniques such as JA3S or Markov Chain Fingerprinting \cite{gancheva2020tls} have since been developed for the purpose of active threat mitigation. However, their efficacy (as well as that of IDS/IPS) is significantly diminished when applied to encrypted traffic. Despite ongoing efforts towards encrypted traffic classification \cite{akbari2021look,yun2022encrypted,zhang2019stnn}, TLS traffic decryption \cite{radivilova2018decrypting} unequivocally remains the most preponderant solution to this problem, levying a compromise between user privacy and general network security.

In spite of this shortcoming, NGFWs are more relevant than ever due to the ongoing adoption of Zero Trust \cite{stafford2020zero} network architectures.
As enterprises seek to implement fine-grained, least-privilege resource access policies, the utilization of micro-segmentation \cite{basta2022towards} has become increasingly common, as shown in a recent report \cite{gonccalves2023beyondcorp} on Alphabet's transition to the Zero Trust paradigm, i.e. Google BeyondCorp \cite{ward2014beyondcorp,osborn2016beyondcorp}. We argue that current access decision systems which are commonly based on user authentication are insufficient for preventing attacks that are facilitated by vulnerable versions of common tools or shared libraries, e.g. the recent \texttt{liblzma} backdoor or software supply chain attacks \cite{ohm2020backstabber}. This problem is further exacerbated by the increasing fluidization of network demarcations, a phenomenon determined by the necessity of free movement between enterprise-owned and public networks. We believe that meeting these challenges should be aided by end-user devices.


\subimport{./}{01-objectives.tex}
\subimport{./}{02-contributions.tex}
\subimport{./}{03-structure.tex}


