\chapter{Evaluarea opțiunilor IP/TCP/UDP peste rețele publice}

Pe măsură ce internetul evoluează, la fel cresc și așteptările pentru noile protocoale de rețea. Indiferent dacă aceste protocoale se referă la criptarea traficului, autentificarea sursei pachetelor sau optimizarea rutării, toate se confruntă cu același obstacol inițial: incertitudinea conformității lor cu numeroasele politici de filtrare utilizate de aproape tot atâtea dispozitive intermediare (middleboxes). Această incertitudine a fost doar exacerbată de ossificarea continuă a protocoalelor fundamentale ale internetului. Un indicator imediat evident al acestei presupuse înrădăcinări este adoptarea stângace a noilor standarde și acceptarea implementărilor conforme cu specificațiile bine stabilite. Aducem în atenție RFC 7126, care oferă recomandări privind filtrarea pachetelor cu opțiuni IPv4 din cauza utilizării lor incorecte pe scară largă. Necesitatea acestuia, după mai bine de treizeci de ani de când mecanismul de extensie IP a fost prevăzut în RFC 791, este emblematică pentru problema în cauză. Indiferent de cauzele care au condus la această situație, obiectivul nu este neapărat de a remedia, ci de a ocoli obstacolul pe care îl reprezintă.

În acest capitol, încercăm să abordăm ambele probleme menționate anterior. Ca răspuns la prima propunere, realizăm o investigație asupra acceptării opțiunilor de protocol pe internet și stabilim viabilitatea lor ca mijloace pentru etichetarea pachetelor. Cea de-a doua problemă o abordăm printr-un studiu al stadiului actual al tehnicii. Acest studiu are scopul de a informa cititorul despre provocările utilizării testării automate pentru stabilirea încrederii în versiuni nedeterministe ale aplicațiilor de rețea.

\section{Prezentarea problemei}

În acest capitol, încercăm să abordăm problema ossificării protocoalelor. Astfel, realizăm o investigație asupra acceptării opțiunilor de protocol pe internet și stabilim viabilitatea lor ca mijloace pentru etichetarea pachetelor. În acest scop, propunem următoarele întrebări de cercetare:

\textbf{RQ1: Este starea actuală a Internetului adecvată pentru opțiunile de protocol deja standardizate?}
Protocoalele existente au fost concepute pornind de la presupunerea că anteturile de lungime fixă vor deveni la un moment dat insuficiente pentru a încapsula informațiile necesare mecanismelor suplimentare. În consecință, majoritatea au inclus o funcție de opțiuni cu lungime variabilă, care permite integrarea acestora pe măsură ce apare nevoia. Beneficiile sunt evidente atunci când observăm modul în care TCP este consolidat prin opțiunea de Scalare a Ferestrei (care crește dimensiunea maximă a ferestrei de la 64Kb la 1Gb) sau opțiunea de Confirmare Selectivă (care permite specificarea unui interval finit de pachete pierdute). În absența acestora, TCP nu ar mai fi viabil în medii cu latență ridicată și lățime de bandă mare, cum ar fi rețelele WAN. Deși inițial opțiunile au fost trecute cu vederea în favoarea redefinirii câmpurilor existente pentru a se potrivi mai bine scopurilor specifice ale anumitor utilizatori, această abordare a fost descurajată ferm de organizațiile de standarde deschise și de comunitate în general. În ciuda cererii crescânde pentru extinderea protocoalelor actuale, Sistemele Autonome (AS) s-au dovedit rezistente la această schimbare. Deși unele opțiuni au fost standardizate, acceptarea lor rămâne extrem de dependentă de producătorii de dispozitive intermediare (middleboxes).

\textbf{RQ2: Pot fi atașate noi opțiuni de protocol unui pachet fără a compromite conformitatea acestuia cu schemele de filtrare?}
Când sistemele de extensie a protocoalelor menționate au fost definite pentru prima dată, o serie de identificatori de opțiuni (adică: puncte de cod, tipuri) au fost desemnați pentru funcționalități necesare imediat la acea vreme (de exemplu: opțiunea de Securitate IP sau opțiunea de Autentificare și Criptare UDP). Pentru a administra corect numărul tot mai mare de proiecte independente și pentru a preveni utilizarea neautorizată a punctelor de cod nealocate să devină o practică obișnuită, IANA a rezervat anumite intervale în acest scop. Aceasta, la rândul său, a dus la introducerea ID-urilor experimentale ca metodă de partajare a intervalelor alocate încă limitate, în timpul testării în medii comune. Cu toate acestea, au existat cazuri cunoscute de utilizare nepermisă a anumitor puncte de cod, unele suprapunându-se cu opțiuni stabilite (de exemplu: Timeout-ul Utilizatorului TCP). Având în vedere că aceste opțiuni neînregistrate și experimentale sunt practic necunoscute pentru majoritatea dispozitivelor intermediare, o preocupare justificată este dacă acestea vor fi acceptate fără verificări suplimentare sau pur și simplu eliminate.

\textbf{RQ3: Schema de anotare luată în considerare interferează cu funcționalitatea de bază a vreunui protocol sau aplicație?}
Presupunând că un pachet modificat poate traversa rețeaua fără obstacole, necesitatea de a asigura funcționarea normală a aplicațiilor al căror trafic intră sub incidența schemei testate este de o importanță capitală. În mod ideal, modificarea pachetului este gestionată în afara sferei aplicației în sine, fie în nucleu (de exemplu: Notificarea Explicită de Congestie a DCCP) sau prin deferire către un alt proces din spațiul utilizatorului (de exemplu: manipularea pachetelor tcpcrypt folosind NetfilterQueue). Cu toate acestea, nu este întotdeauna cazul. Utilizarea ocolirii nucleului poate fi justificată în mod corect de limitări practice, cum ar fi furtunile IRQ în timpul atacurilor DDoS (cea mai mare parte a timpului procesorului este folosită pentru a primi pachete, nu pentru a le procesa). În timpul unor astfel de atacuri, iptables atinge o stare de saturație la aproximativ 1 milion de pachete pe secundă (pps). În timp ce soluțiile bazate pe eBPF sunt cunoscute pentru capacitatea de a elimina până la 15 milioane de pachete pe secundă și de a se recupera rapid dintr-o stare de eliminare a cozii FIFO, tehnologia este încă nouă și impune constrângeri stricte dezvoltatorului. Între timp, soluțiile bazate pe cadre precum Intel DPDK sau PF_RING sunt mai susceptibile de a fi incompatibile cu orice schemă de anotare nouă. În consecință, este necesar un mediu de testare la scară largă pentru a valida orice adiție nouă la protocoalele bine stabilite.

Contribuțiile aduse spre rezolvarea acestor probleme sunt următoarele:
\begin{itemize}
    \item Oferim un instrument \footnote{Cod sursă disponibil la \url{https://github.com/RaduMantu/ops-inject}} capabil să intercepteze pachete specifice și să le anoteze cu tipuri de opțiuni specificate de utilizator, pe protocol. Opțiunile sunt generate în întregime de un decodor care permite integrarea ușoară a noi tipuri de opțiuni, precum și a noi protocoale.
    \item Propunem un cadru pentru evaluarea conformității unui sistem de anotare cu politicile de firewall și furnizăm scripturi de configurare și șabloane pentru experimente în cloud sub principalii furnizori disponibili (i.e.: Google Cloud, AWS, Microsoft Azure, Digital Ocean).
    \item Evaluăm mecanismele actuale de extindere a protocoalelor de nivel 3 și nivel 4 și decidem dacă reprezintă bază stabilă pentru dezvoltarea de noi extensii.
\end{itemize}

\section{Arhitectură}

\subsection{Utilitar pentru adnotarea pachetelor}

În testarea diferitelor caracteristici ale protocoalelor de nivel 3 și nivel 4, o abordare obișnuită constă în generarea de trafic sintetic și ocolirea stivei de rețea a gazdei de origine. Am decis să renunțăm la această metodă în favoarea unei alternative mai practice: modificarea traficului real. Avantajul principal al acestei scheme față de cea anterioară este capacitatea de a verifica nu doar că pachetele annotate pot fi transmise cu succes prin rețea, ci și că modificarea lor nu afectează protocoalele de nivel superior din stiva OSI. În plus, flexibilitatea în evaluarea conformității cu firewall-urile cu stare reduce efortul care ar fi fost necesar pentru a simula sesiuni suficient de credibile.

Aceste aspecte au impulsionat dezvoltarea unei metode de inserare a mai multor opțiuni specifice protocolului în traficul de ieșire. Una dintre caracteristicile esențiale ale instrumentului rezultat este capacitatea de a implementa cu ușurință noi opțiuni pentru protocoalele existente sau de a înregistra noi protocoale în întregime. Limitat de amploarea nevoilor noastre pentru acest experiment, urmărirea sesiunilor se află în afara scopului său. Prin urmare, opțiuni precum MultiPath TCP nu sunt disponibile pentru testare. În schimb, toate opțiunile ar trebui să fie fie de natură statică (de exemplu, NOP), fie să depindă în întregime de informațiile disponibile în antetul și conținutul unui singur pachet (de exemplu, sume de control alternative). În caz contrar, utilizarea sa poate fi limitată la determinarea doar dacă pachetul inițial a reușit să traverseze rețeaua (de exemplu, marcajele de timp TCP).

În continuare, prezentăm o descriere sumară a funcționării instrumentului. Inițial, utilizatorul adaugă una sau mai multe reguli \texttt{iptables} cu un target \texttt{NFQUEUE}. Toate pachetele care corespund uneia dintre aceste reguli sunt redirecționate în spațiul utilizatorului pentru ca instrumentul nostru să le evalueze și să le modifice după cum consideră de cuviință. La invocarea inițială a instrumentului, utilizatorul poate specifica o secvență de octeți care reprezintă puncte de cod ale opțiunilor. Pentru fiecare pachet, punctele de cod vor fi extinse la intrări TLV reale și inserate într-o secțiune de opțiuni (potențial noi). Callback-ul NetfilterQueue care îndeplinește acest scop urmează trei pași pentru o anotare reușită. În continuare, descriem acești pași în detaliu:

\textbf{Decodificarea Opțiunilor:} Având acces la pachet, așa cum este furnizat de cârligul kernel Netfilter, primul pas constă în generarea conținutului secțiunii de opțiuni pentru un protocol prestabilit (cum ar fi IP, TCP, UDP). Utilizând secvența de octeți de tip opțiune specificată de utilizator, sunt apelate funcții decodoare specifice pentru a le extinde, populând astfel buffer-ul de opțiuni. Această fază de extindere este, în mare parte, un proces arbitrar. De exemplu, o opțiune de marcaj temporal IP (\texttt{0x44}) va fi tradusă într-un buffer de 12 octeți care conține un singur marcaj temporal asociat adresei IP sursă. Steagurile generate vor indica câmpuri de adresă prestabilite pentru a descuraja adăugarea de noi marcaje temporale de către dispozitivele intermediare conforme. Această limitare a controlului utilizatorului este o decizie conștientă menită să simplifice utilizarea instrumentului. Cu toate acestea, funcțiile decodoare pot fi modificate cu ușurință de la caz la caz pentru a accepta mai multe informații și pentru a se conforma cerințelor mai detaliate. Trebuie notat că opțiunile vor apărea în ordinea în care au fost specificate inițial de utilizator. Totuși, ordinea în care sunt generate poate varia. De exemplu, o sumă de control pentru opțiunile UDP (\texttt{0x02}) ar trebui plasată cât mai aproape de începutul secțiunii, în timp ce necesită ca toate celelalte opțiuni să fi fost deja calculate. În consecință, decodarea sa este amânată pe baza unei atribuiri discreționare a priorității opțiunilor.

\textbf{Reasamblarea Pachetelor:} Acest pas modifică pachetul inițial pentru a include noile opțiuni generate. Toate câmpurile asociate (e.g., \texttt{IP.total\_length}), cu excepția sumei de control, sunt ajustate pentru a ține cont de decalajul încărcăturii. În funcție de preferințele utilizatorului, opțiunile existente pot fi fie înlocuite complet, fie păstrate. În testele noastre, am adoptat prima abordare pentru a exercita un control deplin asupra conținutului secțiunii de opțiuni. Alternativ, opțiunile existente ar avea prioritate. La rândul său, acest lucru ar putea duce la eliminarea opțiunilor specificate de utilizator care depășesc capacitatea din cauza constrângerilor de spațiu.

\textnf{Recalcularea Checksum-ului:} Motivul pentru care acest pas este separat de cel anterior este că modificările aduse unui antet corespunzător unui nivel inferior pot afecta suma de control a unui protocol de nivel superior. De exemplu, deși opțiunile IP sunt destinate să extindă doar antetul IP, lungimea totală a pachetului crește totuși. Ca urmare, pseudo-antetul TCP sau UDP se modifică, iar suma de control de nivel 4 se schimbă odată cu acesta.
