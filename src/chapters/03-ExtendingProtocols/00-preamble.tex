As the Internet evolves, so do the expectations for new network protocols. Whether said protocols pertain to traffic encryption, packet source authentication or routing optimization, all are faced with the same initial hurdle: the uncertainty of their compliance with the myriad filtering policies employed by virtually just as many middleboxes. This incertitude has only been exacerbated by the continual ossification of the foundational internet protocols. An immediately apparent indicator of this suggested entrenchment is the maladroit adoption of new standards and acceptance of implementations conforming to well-established specifications. We draw attention to RFC 7126 \cite{RFC7126}, offering recommendations on filtering packets with IPv4 options due to widespread misuse. Its necessity after upwards of thirty years since this IP extension mechanism was provisioned in RFC 791 \cite{RFC0791} is epitomic of the issue at hand. Regardless of the underlying causes that lead to this predicament, the objective is not necessarily to remedy but to circumvent the obstacle that it poses.

