As the Internet evolves, so do the expectations for new network protocols. Whether said protocols pertain to traffic encryption, packet source authentication or routing optimization, all are faced with the same initial hurdle: the uncertainty of their compliance with the myriad filtering policies employed by virtually just as many middleboxes. This incertitude has only been exacerbated by the continual ossification of the foundational internet protocols. An immediately apparent indicator of this suggested entrenchment is the maladroit adoption of new standards and acceptance of implementations conforming to well-established specifications. We draw attention to RFC 7126 \cite{RFC7126}, offering recommendations on filtering packets with IPv4 options due to widespread misuse. Its necessity after upwards of thirty years since this IP extension mechanism was provisioned in RFC 791 \cite{RFC0791} is epitomic of the issue at hand. Regardless of the underlying causes that lead to this predicament, the objective is not necessarily to remedy but to circumvent the obstacle that it poses.

Another fundamental issue that we need to address prior to introducing our firewalling solution is that of defining of policies based on application versioning. Our solution was meant to function primarily with applications installed via a package manager. In case of a supply chain attack, our firewall policies would compare the public digests of the packaged files with those of the stored binaries. This model lends itself well to the version control enforced by our firewall. However, locally built OSS or proprietary software can present any number of variations in the absence of deterministic compilation. Allowing them access to the network must be done on a case-by-case basis that requires additional investigation. Although we cannot offer generic guidelines to establish such trust, we explore the possibility of using network fuzzing as a form of automated testing. This process has already been streamlined to a certain degree with the introduction of DeepState \cite{goodman2018deepstate}, a tool capable of integrating fuzzing and symbolic execution into unit testing frameworks. Nonetheless, fuzzing network applications poses a number of unique challenges that have not been fully explored.

In this chapter we try to address both of the aforementioned problems. In reponse to the former proposition, we perform an investigation of protocol options acceptance in the Internet and establish their viability as conduits for the purpose of packet tagging. The latter of the two problems we address by means of a state of the art survey. This survey is meant to inform the reader on the challenges of emplying automated testing for establishing trust in non-deterministic builds of network applications.
