\section{Protocol extension mechanisms}
\label{background:protocols}

%%%%%%%%%%%%%%%%%%%%%%%%%%%%%%%%%%%%%%%%%%%%%%%%%%%%%%%%%%%%%%%%%%%%%%%%%%%%%%%%
\subsection{IP options}

In RFC 791 \cite{RFC0791}, the authors added an \textit{optional} variable-length
section to the IPv4 specification. This section would normally be unnecessary but
could provide useful control functions in certain situations. Its presence would
be detected via an Internet Header Length (IHL) field value greater than 5. The
IHL field represents the number of double-words (i.e., 32-bit structures) that
comprise the IP header and is defined as a 4-bit bit field. As a result, its
maximum value is 15, meaning that the options section could at most measure 40
bytes in size.

This variable-sized addition to the IP header represents a buffer where a list
of Type-Length-Value (TLV) tuples is stored. Each unique option is assigned a
1-byte codepoint for identification purposes; that is the Type. The Length is
another 1-byte field that specifies the length of that option, including these
first two fields. Finally, the Value is a buffer that is used arbitrarily by
the TCP/IP stack that imeplements the option. There are some exceptions to this
rule. Some options do not have a Value field, or even a Length field if that
length is constant and standardized. For example, the End of Options List (EOL)
option that indicates the end of the section, or the No Operation (NOP) option
used strictly as padding for alignment purposes.

Although the original IPv4 RFC states that \textit{"They must be implemented by
all IP modules (host and gateways)"} and \textit{"What is optional is their
transmission in any particular datagram, not their implementation"}, their
unequivocal acceptance could pose security threats. For example, the inclusion
of the Strict Source and Record Route (SSRR) option could be used for probing
the topology of the target network. Similarly, a large number of NOPs could be
used as a denial of service attack in that they exhaust the CPU resources of the
router by forcing it to parse up to 40 inconsequential options per packet. For
this reason, any sequence of 4 or more consecutive NOPs is usually classified as
malicious.

%%%%%%%%%%%%%%%%%%%%%%%%%%%%%%%%%%%%%%%%%%%%%%%%%%%%%%%%%%%%%%%%%%%%%%%%%%%%%%%%
\subsection{TCP options}

TCP options are similar to their Layer 3 counterpart in that they had also been
standardized in their intial RFC release \cite{rfc793}. In contrast to IP options
however, these have since known wider adoption and community interest. While IP
options have been relegated to infrastructure debugging purposes with occasional
attempts of extending their usefulness in the security domain (e.g., the
Commercial IP Security Option for packet labeling), TCP options became
indispensable for achieving acceptable performances in WANs.

Structurally, TCP options function the same as IP options. Instead of the IHL
field, the size of the TCP header is determined by the Data Offset field (also
4-but wide). The contents of the variable-length TCP options section is also a
list of TLV tuples. One distinction however, is in the methodology of
prototyping new options in the Internet. The Internet Assigned Numbers Authority
(IANA) assigned a special codepoint for both Experimental IP (30) and TCP (253,
254) options. However, the disproportionate interest in the development of new
TCP options lead to potential collisions in public networks. As a result, the
specification was updated in RFC 6994 with the introduction of the Shared Use of
Experimental TCP Options. Now, experimental options must contain a 16-bit or
32-bit Experiment ID (ExID) field that is registered with IANA.

%%%%%%%%%%%%%%%%%%%%%%%%%%%%%%%%%%%%%%%%%%%%%%%%%%%%%%%%%%%%%%%%%%%%%%%%%%%%%%%%
\subsection{UDP options}

UDP options are an oddity in that they were not part of the initial design
\cite{postel1980rfc0768}. In fact, UDP options are not even standardized at the
time of writing and are still an active IETF draft almost a decade after being
first proposed in July 2015.

Although there is no equivalent to the IHL or Data Offset fields, UDP options
can still exist due to a redundancy between the IP Total Length and UDP Length
fields. By increasing the former while leaving the latter unmodified, one can
create a trailer section after the UDP payload. One benefit of this approach is
that the UDP section length is limited only by the maximum IP packet size. This
difference may seem insignificant but the 40-byte size limit in TCP options is
the sole reason why no new options can be developed today; most of the options
section space has been exhausted by other, more critical options. Considering
the push towards QUIC \cite{langley2017quic} and away from TCP/TLS, this can
lead to renewed interest in the development of new experimental Transport Layer
options.

Nonetheless, UDP options are still insufficiently tested in the wider Internet.
Moreover, different implementations of the UDP protocol may erroneously ignore
the UDP Length field and calculate the payload size strictly based on the
IP Total Length, leading to a breakdown in communication at the Application
Layer.

