During the past twenty odd years, each industrial firewall manufacturer
implemented certain features derived from the Next Generation Firewall paradigm.
Some problems such as traffic decryption are considered to be solved and all
solutions are similar in how they handle it. Others are of more immediate interest
and the answer is not necessarily obvious. In this chapter we focus on
understanding how the current application identification features of modern
firewalls function. In what ways are they similar, and how do they distinguish
themselves from one another. In most cases, the mechanisms in play are the
intellectual propriety of the OEMs and the software is closed source. Thus, much
of the information is derived from technical reports and product briefs.

Another concept that is fundamental to this thesis are network protocol options.
In our work, we have developed a traffic annotation scheme that allows us to
tag packets that comply with the firewall policy of the originating host. We
consider it paramount that our annotation scheme does not break communication
with systems that are unaware of it. As such, we decided to use the protocol
extension mechanisms that have (in most cases) been in place since the creation
of the protocol itself. In this chapter we explain how IP, TCP and UDP options
work. However, we do not discuss their level of adoption in the wider Internet
or their status within their respective IETF working groups. This chapter is
intended to provide a basic understanding of how these mechanisms function.
Other aspects are discussed at length in the following chapters.

