\section{Future Work}
\label{conclusion:future}

At the time of writing, we are pursuing two additional goals that are related
to this thesis:

\begin{enumerate}
    \item \textbf{Netfilter Queue bug:} When developing the \texttt{iptables}
    plugin for verifying the SHA256-HMAC calculated over the Layer 4 payload
    and stored as an experimental IP options, we noticed that the \texttt{tcp_hdr()}
    macro returned an incorrect pointer. Not accounting for the newly added 36
    byte IP options section, it instead returned a pointer memoized before
    reaching the \texttt{OUTPUT} chain. At the moment, we are investigating
    uses of macros or inline functions that act as wrappers over
    \texttt{skb_transport_header()} (similar to \texttt{tcp_hdr()}) that are
    used /textit{after} the \texttt{OUTPUT} chain verdict is established. Our
    aim is to determine the level of criticality of this bug, and then patch it.

    \item \textbf{Runtime verification of JIT-ed code:} The most consistent
    criticism we have faced so far relates to our decision to not include
    Just In Time compier output in our process identity measurements. At the time
    we argued that this problem is not trivial and should be studied separately.
    The continuous changes that JIT engines perform on the anonymous executable
    memory ranges that they manage make it impossible to perform accurate
    measurements, or compare these measurements with any existing database of
    legitimate applications. At this time, we are exploring the possibility of
    implementing a Linux Security Module that may block \texttt{mmap()} and
    \texttt{mprotect()} calls based on the evaluation of a userspace process
    that can perform mapware fingerprint scanning with YARA \cite{naik2020embedding}
    and other forms of static anomaly detection.
\end{enumerate}
