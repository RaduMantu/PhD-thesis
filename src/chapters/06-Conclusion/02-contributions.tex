\section{Contributions}
\label{conclusion:contrib}

In support of this thesis, we claim a number of contributions that address
emerging issues with firewall technologies. These are as follows:

\begin{enumerate}
    \item \textbf{Evaluation of protocol option acceptance in the Internet:}
    We implemented \texttt{ops-inject} \cite{mantu2024framework}, a network
    traffic annotator based on Netfilter Queue that can attach arbitrary
    interpretations of IP, TCP or UDP options to outgoing packets. We then
    utilized this tool to perform all-to-all reachability tests between 21
    hosts with a diverse geographic distribution, from moultiple cloud providers.
    This study helped clarify the current state of options acceptance in the
    Internet and informed our choice of conduit for network traffic tagging.

    %% TODO: add citation to AppScout when published 
    \item \textbf{Implementation of an application-aware distributed firewall:}
    We implemented \daf{} \cite{mantu2024process} and \scout{},
    two state of the art firewalls capable of filtering traffic based on the
    identity of the endpoint application without performing deep packet inspection
    or performing traffic decryption. Through these two alternatives, one
    bound to user space and the other designed as a kernel module, we attain near
    linereate throughputs while also accounting for microservice architecutres
    by supporting Linux namespace transparency.

    \item \textbf{Evaluation in realistic environments:} We tested our two Linux
    variants on both desktop systems with 1Gbps NICs, as well as datacenter
    environments with 10Gbps NICs. We performed an in-depth inspection of Netfilter
    Queue, an entrenched kernel mechanism that allows IDS and IPS to perform
    deep packet inspection in userspace. Based on our observations and experience
    researching this topic \cite{gherghescu2024ve}, we propose a number of
    improvements that could further increase its performance and bridge the gap
    between a kernel-only solution and a userspace firewall. Additionally, we
    identify potential security risks that this technology may unknowingly introduce.

    \item \textbf{Packet annotation scheme and compliance verification:}
    We propose a packet annotation scheme based primarily on IP options. This
    traffic tagging mechanism is fully implemented in \daf{} and serves to
    enable middleboxes to verify the compliance of individual pakets with the
    firewall policies of its originating system. We also provide a Windows
    prototype \cite{gherghita2024label} that serves to demonstrate the
    portability of our solution and the possibility of utilizing TCP options as
    a conduit in restrictive networks where IP options are blocked.

    \item \textbf{Network integration solutions:} In order to facilitate easier
    adoption, we designed \daf{} with as few deployment requirements as
    possible. This includes a lack of kernel components, making it an ideal
    choice for systems employing the lockdown LSM. Additionally, we provide
    a \texttt{snort3} plugin for matching and verifying HMAC annotations as
    IP options, as well as \texttt{iptables} plugins serving the same purpose
    for both the IP option conduit, and for the TCP option conduit. These
    extensions allow users to deploy our firewall on a small subset of hosts and
    perform on the path compliance verification using conventional tools.

    \item \textbf{Overview of the state of the art in Network Fuzzing:}
    By applying our previous experience in fuzzer development \cite{nikolic2021refined},
    we explore the current challenges of employing network fuzzing
    \cite{mantu2023network} as a method of reinforcing the trust placed in a
    whitelisted application. This serves as our recommendation for adopters of
    \daf{} that need to deploy their own proprietary solutions or custom builds
    of OSS without deterministic compilation.
\end{enumerate}
